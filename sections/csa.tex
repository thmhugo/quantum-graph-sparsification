\section{Classical Sparsification Algorithm}
\label{seq:classical_sparsification}

The classical algorithm for graph sparsification is based on edge sampling,
where each edge is added to the sparsifier according to a fixed probability
distribution.

In order to be sure $L_H$ effectively approximates $L_G$, the choice of each
$p_e$ cannot be done at random. A nearly-linear classical sparsification
algorithm was introduced by Spielman and Srivastava
\cite{spielman_graph_2011}, by approximating effective resistance
between any two edges in the graph efficiently, and thus introducing a way to
correctly sampling the edges.

\subsection{Effective resistance}
% If the graph is interpreted as an electrical network, the effective
% resistance is the

% The effective resistance of an edge is a value that is associated with it,
% which can be seen as , when $G$ is interpreted as an electrical network.

In the case of unweighted graphs, the effective resistance of an edge can be
related to the connectivity of the graph: edges belonging to strong components
have a low effective resistance, and vertex cut (whose removal renders G
disconnected) tends to have a high effective resistance.

\begin{Figure}
    \centering
    \begin{tikzpicture}[scale=0.25]
        \coordinate (right) at (8,1) {}; \coordinate (left) at (-8, 1) {};

        \begin{scope}[shift={(right)},rotate=30,transform shape]
            \graph [nodes={draw,circle,as=}]
                { subgraph K_n [name=right,n=6,clockwise,radius=5cm] };
        \end{scope}
        \begin{scope}[shift={(left)},rotate=-30,transform shape]
            \graph [nodes={draw,circle,as=}]
                { subgraph K_n [name=left,n=6,clockwise,radius=5cm] };
        \end{scope}
        \draw[line width=2pt] (right 6) -- (left 2);
      \end{tikzpicture}
    \captionof{figure}{Graph illustrating edge of high effective resistance.}
\end{Figure}

The effective resistance of the bold edge is roughly $r_e = 1$, while the
effective resistance of the other is $r_e \in O(\frac{1}{n})$. Spielman and
Srivastava associated then the probability for an edge to be kept while
constructing the sparsifier proportionally to $r_e \omega_e$.

In a graph $G=(V,E)$ the effective resistance $r_e$ of an edge $e=(u,v)$, with $u,v$ two nodes,
can be expressed with the quadratic form
\begin{equation}\label{eq:effective-resistance}
   R_{u,v} = (\x_u - \x_v)^TL_G^+(\x_u - \x_v) \text{ ,}
\end{equation}
where $\x_i$ is the $i^{th}$ vector of the canonical basis. \cite{spielman_graph_2011}


\subsection{Graph spanner}
The \textit{distance} between two nodes $u$ and $v$ with respect to $G$ is
defined as:
\begin{equation*}
    \delta_G(u,v) = \min_{u\rightarrow v} \sum_{i} \omega(p_{i}, p_{i+1})^{-1},
\end{equation*}
which is consistent with the previous definition of effective resistance of an
edge, where $\{ u\rightarrow v\}$ is the set of all paths from $u$ to $v$ in
$G$, each element $u\rightarrow v = \{p_0, \cdots, p_k\}$ is a set of vertices
of $V$. A spanner $H$ of a graph $G$ is a subgraph of $G$ with fewer edges,
where a trade-off is made between the number of edges and the stretching of
distances.
\begin{definition}
    An $(\alpha,\beta)$-spanner of the graph $G=(V,E)$ is a subgraph $H = (V,
    E_H)$ with $E_H \subseteq E$, such that $\forall u, v \in V$,
    \begin{equation*}
        \delta_G(u,v) \leq \delta_H(u,v) \leq \alpha\delta_G(u,v) + \beta.
    \end{equation*}
\end{definition}
This definition holds for weighted graphs, in which case the weight of the kept
edges stay unchanged. In the following only multiplicative spanners are
considered, i.e. $\beta = 0$ and $\alpha = 2 \log n$, namely, $2\log
n$-spanners. Furthermore, key objects of the algorithm for graph
$\epsilon$-spectral sparsification described below are $r$-packings spanners.
\begin{definition}\label{def:packing-spanner}
    Let $G$ be a graph, an $r$-packings spanner of $G$ is an ordered set $H=(H_1,
    \cdots, H_r)$ of $r$ edge-disjoint subgraphs of $G$ such that $H_i$ is a
    spanner for the graph $G - \bigcup_{j=1}^{i-1} H_j$.
\end{definition}

%One can notice that $G - \bigcup_{j=1}^{i-1} H_j$ is the remaining graph after
%removal of the edges of all previous spanners.

Koutis and Xu proposed the following algorithm \cite{koutis_simple_2016},
using the effective resistance of each edge as exhibited by Spielman and
Srivastava to construct $t$-packings spanner of the input graph:
% Sparsification by edge sampling algorithm
\begin{algorithm}[H]
    \caption{\textbf{ClassicallySparsify}($G,
    \epsilon$)}\label{alg:classical-sparsify}
    \begin{algorithmic}[1]
    \State \textit{construct an $O(\frac{\log n}{\epsilon ^2})$-packing spanner
    $H$ of $G$}
    \State $\tilde{G} \< H $ \ForAll{$e \notin H$} \\
    \textit{$\;$ w.p. $\frac{1}{4}$ add e to $\tilde{G}$, with weight
    $4\omega_e$} \EndFor \\
    \Return $\tilde{G}$
    \end{algorithmic}
\end{algorithm}
and provided the following theorem:
\begin{theorem}[Classical sparsifier]\label{thm:classical-sparsifier} The output
    $\tilde{G}$ of \textbf{ClassicallySparsify} on inputs $G$ and $\epsilon$
    satisfies with probability $1 - \frac{1}{n^2}$
    \begin{equation*}
        (1 - \epsilon) L_G \preccurlyeq L_{\tilde{G}} \preccurlyeq (1 + \epsilon) L_G
    \end{equation*}
    Moreover, the expected number of edges in $\tilde{G}$ is at most
    $\tilde{O}(\frac{n}{\epsilon ^2} + \frac{m}{2})$.
\end{theorem}

The proof of \autoref{thm:classical-sparsifier}
\cite{koutis_simple_2016} ensures that the output of
\textbf{ClassicallySparsify} is an $\epsilon$-spectral sparsifier. A single
iteration of the above procedure divides the number of edges in the output graph
by roughly two. Hence, repeating $ t \in O(\log \frac{m}{n})$ times
\textbf{ClassicallySparsify($G, \epsilon'$)} with $\epsilon' \in
O(\frac{\epsilon}{t}) $ results in an $\epsilon$-spectral sparsifier with
$\tilde{O}(\frac{n}{\epsilon ^2})$ edges.

Complexity-wise, the execution time of the provided algorithm is mostly
dominated by the construction of the $\tilde{O}(\frac{1}{\epsilon^2})$ spanners,
each of which requires time $\tilde{O}(m)$, giving a total time complexity of
$\tilde{O}(\frac{m}{\epsilon^2})$.